
\section{Objetivo Geral} % (fold)
\label{sub:objetivo_geral}
	
	Algumas pessoas costumam limpar suas residências de forma regular e outras que por motivos diversos só fazem isso esporadicamente. O desenvolvimento do R2-PI2 tem como alvo atender aos usuários que desejam manter o local limpo de modo automático e com o mínimo de intervenção humana. Alguns aspiradores robôs já fazem a limpeza do ambiente automaticamente, porém uma parcela deles necessita que o usuário tenha que procurá-los pela casa enquanto eles emitem um sinal sonoro indicando que a bateria está quase descarregada. O R2-PI2 será apto a voltar sozinho para a base e permanecer lá até que a bateria seja carregada, além de manter-se inativo na base ele poderá voltar a efetuar a higienização do local quando programado pelo utilizador. Outros objetivos que devem ser atendidos pelo dispositivo aspirador, estão listados no item seguinte.
	Na figura \ref{img:eap} está apresentada a Estrutura analítica do projeto – EAP, onde destacam-se as entregas e seus subsistemas ao longo do projeto.

% subsection objetivo_geral (end)

\subsection{Objetivos Específicos} % (fold)
\label{sub:objetivos_específicos}
	
	\begin{itemize}
		\item Controle a distância;
		\item Se locomover pelo cômodo e desviar de obstáculos;
		\item Comunicação entre o Robô e a central de processamento;
		\item Identificar e sinalizar status da bateria;
		\item Possuir sistema de acompanhamento do status do robô
		\item Conter acionamento de start/stop;
		\item Gerar relatório de atividades;
		\item Sistema de agendamento de limpezas.
	\end{itemize}
% subsection objetivos_específicos (end)
