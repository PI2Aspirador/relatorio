\section{Equipe e Responsabilidades} % (fold)
\label{sub:equipe_e_responsabilidades}

	A equipe do projeto está distribuida de acordo com a Tabela \ref{tab:equipe}. Ou seja, é uma equipe formada por 14 (catorze) estudantes, distribuidos em 5 (cinco) engenharias distintas, o que torna a gestão e distribuição do conhecimento uma tarefa bastante complicada.

	\begin{table}[H]
	\centering
	\caption{Equipe do projeto R2-PI2}
	\label{tab:equipe}
	\begin{tabular}{|c|c|c|}
	\hline
	\textbf{Nome}                              & \textbf{Matricula} & \textbf{Engenharia} \\ \hline
	Kaio Diego de Araujo Coelho                & 12/0123673         & Eletrônica          \\ \hline
	Laryssa Lorrany Olinda Costa               & 12/0060973         & Eletrônica          \\ \hline
	Mônica Damasceno Cavalcante Castelo Branco & 10/0037097         & Eletrônica          \\ \hline
	Rafael Fazzolino Pinto Barbosa             & 11/0136942         & Software            \\ \hline
	Ricardo Gonçalves Teixeira                 & 12/0021561         & Software            \\ \hline
	João Paulo Siqueira Ribeiro                & 12/0014378         & Software            \\ \hline
	Thais Soares Monteiro                      & 11/0066561         & Energia             \\ \hline
	Pedro Henrique de Queiroz Rocha            & 11/0083890         & Energia             \\ \hline
	Jair Jorge Medeiros                        & 11/0013760         & Energia             \\ \hline
	Hildoglas Botelho Chaves                   & 11/0121104         & Energia             \\ \hline
	Luan de Oliveira Noleto                    & 11/0128419         & Energia             \\ \hline
	Rafael de Souza Freitas                    & 11/0019300         & Automotiva          \\ \hline
	Yago Henrique Melho Honda                  & 12/0042840         & Aeroespacial        \\ \hline
	Márcia Aline Ribeiro Silva                 & 12/0017806         & Aeroespacial        \\ \hline
	\end{tabular}
	\end{table}

	Como estamos trabalhando com áreas de conhecimento distintas, vê-se a necessidade de buscar maneiras que possam garantir maior equilíbrio de conhecimento na equipe, seja relacionado ao processo de desenvolvimento ou à áreas específicas de cada sub-sistema. Com este objetivo, adotamos o papel de \textit{Scrum master} como um facilitador durante as \textit{sprints} de desenvolvimento.

	Entretanto, a simples adoção deste papel pode não garantir a distribuição do conhecimento pela equipe, desse modo optou-se por rotacionar o papel de \textit{scrum master} entre os integrantes. Fazendo com que cada integrante tenha a oportunidade de vivenciar esta experiência pelo período de uma \textit{sprint}, ou seja, 2 (duas) semanas.

	A distribuição das responsabilidades em relação ao desenvolvimento da solução estão de acordo com a Tabela \ref{tab:areas}, onde estão destacados os sub-sistemas presentes na solução.

	% Please add the following required packages to your document preamble:
% \usepackage{multirow}
\begin{table}[H]
\centering
\caption{Equipe - Áreas de atuação}
\label{tab:areas}
\begin{tabular}{|c|c|}

\hline
\multirow{2}{*}{\textbf{Eletrônica}}                                                            & Sensoriamento               \\ \cline{2-2} 
                                                                                                & \multirow{2}{*}{Comunicação} \\ \cline{1-1}
\multirow{3}{*}{\textbf{Software}}                                                              &                              \\ \cline{2-2} 
                                                                                                & Navegação                    \\ \cline{2-2} 
                                                                                                & Interface                    \\ \hline
\textbf{Energia}                                                                                & Alimentação                  \\ \hline
\multirow{3}{*}{\textbf{\begin{tabular}[c]{@{}c@{}}Automotiva\\ e\\ Aeroespacial\end{tabular}}} & Estrutura                    \\ \cline{2-2} 
                                                                                                & Locomoção                    \\ \cline{2-2} 
                                                                                                & Sucção                       \\ \hline
\end{tabular}
\end{table}

% subsection equipe_e_responsabilidades (end)
\subsection{Política de Comunicação da Equipe} % (fold)
\label{sub:política_de_comunicação_da_equipe}
	
	Para a comunicação do grupo foram escolhidas algumas ferramentas de comunicação para facilitar a organização do grupo. De maneira informal foi adotado o aplicativo \textit{Whatsapp}. Como foi adotado o \textit{Scrum}, para o gerenciamento do projeto, está sendo utilizado o \textit{Trello}\footnote{www.trello.com} para organizar todas as atividades a serem feitas e registrar as que já estão prontas, utilizando a técnica de Kanban. Para compartilhar e editar documentos para os pontos de controle foi criada uma pasta no \textit{Google Drive}\footnote{www.drive.google.com}. 

	A última ferramenta para a comunicação do grupo é a plataforma \textit{Slack}\footnote{www.slack.com}, que funciona como um chat, podendo criar canais de comunicação que apenas os interessados no assunto entram no canal facilitando a troca de informações sobre determinado assunto ou sobre determinada parte do projeto. Além do canal geral com todos os integrantes onde ocorre a discussão com o grupo todo, um canal criado no \textit{Slack} e bastante importante é o \textit{daily}, onde cada um deve, diariamente, descrever o que fez no dia para o Projeto, para facilitar o controle do grupo. 
