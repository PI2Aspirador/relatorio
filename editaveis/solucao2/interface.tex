\section{Interface} % fold
\label{sub:interface}
  Para a realização da interface com o usuário, foi criado uma aplicação web, utilizando o \textit{framework} livre, Ruby on Rails\cite{rails}.
Esse \textit{framework} permite desenvolvimento de sites e usa como linguagem o Ruby. Sua arquitetura padrão é o MVC(\textit{Model-View-Controller}),
 que basicamente separa a aplicação em 3 camadas. A camada de interação do usuário(\textit{view}), a camada de manipulação dos dados(\textit{model}) 
e a camada de controle(\textit{controller})\cite{mvc}.
  O rails permite adicionar bibliotecas com funções especias, essas bibliotecas tem o nome de \textit{gem}. Para nossa aplicação, além das gems já padrões para a utilização
  do rails, também utilizamos as gems que podem ser vistas na tabela \ref{gems}.
\begin{table}[H]
\centering
\caption{Gems}
\label{gems}
\begin{tabular}{|l|l|}
\hline
\rowcolor[HTML]{C0C0C0} 
{\color[HTML]{333333} \textbf{Gem}} & {\color[HTML]{333333} \textbf{Descrição}} \\ \hline
devise\_ldap\_authenticatable       & Para autenticar junto com uma base LDAP   \\ \hline
cancancan                           & Para autorização                          \\ \hline
rolify                              & Para criar funções de usuários            \\ \hline
fullcalendar-rails                  & Para mostrar calendário de agendamento    \\ \hline
momentjs-rails                      & Auxílio ao calendário                     \\ \hline
bootstrap-sass                      & Para layout das páginas                   \\ \hline
rails-erd                           & Para gerar diagrama de classe             \\ \hline
\end{tabular}
\end{table}
