\section{Navegação} % (fold)
\label{sec:navegação2}
	
	Como o processamento de todas as informações obtidas durante a navegação ocorrerá na base, sabe-se que o tempo de resposta do servidor é uma variável importante quando se refere a um sistema de tempo real, como o proposto pelo projeto. Dessa maneira, fez-se necessária a implantação do \textit{patch} \textit{rt\_preempt} no kernel do linux presente na \textit{raspberry}. Para isso, utilizamos como fonte de conhecimento a wiki oficial do projeto \textit{RT\_preempt}, disponível \href{https://rt.wiki.kernel.org/index.php/Main_Page}{aqui}.

	Com a configuração e recompilação do kernel com este \textit{patch}, obtivemos um tempo de resposta aproximado de 19 micro segundos, o que foi considerado bom pela equipe do projeto. Uma análise foi feita utilizando o \textit{script} \textit{cylicltest}, da mesma equipe \textit{RT\_preempt}, para calcular o tempo mínimo, médio e máximo de resposta. Na simulação foi utilizado um processo com prioridade 80, com 100.000 (cem mil) \textit{loops} a um intervalo de 500 micro segundos, obtendo o resultado apresentado na Figura \ref{img:tempo_de_resposta}.

	\begin{figure}[H]
		\centering
		\includegraphics[scale=0.7]{figuras/tempo_de_resposta.png}
		\caption{Tempo de resposta do servidor.}
		\label{img:tempo_de_resposta}
	\end{figure}

	Após a implantação deste requisito de tempo real, buscou-se definir uma estratégia de navegação que contemple os requisitos iniciais do projeto. Para isso, o sistema de navegação foi dividido em 2 contextos. A navegação pode estar no contexto de \textit{running}, onde estará rodando pelo ambiente de maneira aleatória, ou \textit{back}, no qual o robô deverá retornar para base.

	Nos sub-tópicos a seguir estão detalhadas as estratégias de navegação nos dois contextos.

	\subsection{Running} % (fold)
	 \label{sub:running}
	 
	 	O algoritmo de navegação utilizado durante este contexto é bastante simples, o qual envolve uma estratégia de navegação aleatória. Basicamente, o robô sempre tenderá a andar para frente, quando for encontrado um obstáculo, o servidor enviará uma ordem para desviar daquele obstáculo, levando em consideração as distâncias laterais do robô. Para isso, o robô utiliza 3 sonares, um apontado para a frente, e os outros dois apontados um para cada lado do robô.

	 	Esta navegação aletarória ocorre até que seja determinado o recuo à base, seja por falta de bateria ou por entrada \textit{stop} por parte do usuário.

	 	A utilização de \textit{encoders} é necessária para minimizar a margem de erro na angulação de curvas e distâncias percorridas. Porém, sua utilização está suspensa durante esta segunda fase do projeto, sendo implementado apenas na terceira etapa, fase de integração dos sub-sistemas.
	 % subsection running (end) 

	 \subsection{De volta à base (Back)} % (fold)
	 \label{sub:de_volta_a_base_}
	 	
	 	O retorno à base deve levar em consideração o tempo disponível de bateria, ou seja, o caminho para base deve ser razoavelmente eficiente. Porém, a utilização da potência de sinal do \textit{wifi} gerou problemas relacionados a precisão destes dados. Por este motivo, o retorno a base também se encontra em estado de implementação. 

	 % subsection de_volta_a_base_ (end)
% section navegação (end)