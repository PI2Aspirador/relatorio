O gerenciamento de riscos em um projeto tem como objetivo orientar a equipe do sobre os riscos presentes, como serão controlados e monitorados, além de aumentar a probabilidade e  impacto de eventos positivos e reduzir a probabilidade e impacto dos eventos negativos.

O processo consiste na realização de um plano de gerenciamento que descreva a análise e execução dos processos de riscos, iniciando-se pela identificação dos mesmos, suas análises quantitativas e qualitativas, plano de respostas e por fim a solução de como eles serão monitorados e controlados durante o ciclo de vida do projeto.

\section{Processo de Gerenciamento de Riscos} % (fold)
\label{sec:processo_de_gerenciamento_de_riscos}
	
	O processo de gerenciamento de riscos nesse projeto, ocorrerá nas seguintes etapas:
	\begin{itemize}
		\item Identificar os riscos e determinar quais deles podem afetar o projeto, documentando suas características.
		\item Realizar a análise qualitativa dos riscos.
		\item Avaliar a exposição ao risco para priorizar os que serão objeto de análise ou ação adicional. 
		\item Realizar a análise quantitativa dos riscos.
		\item Efetuar a análise numérica do efeito dos riscos identificados nos objetivos gerais do projeto.
		\item Planejar as respostas aos riscos, desenvolvendo opções e ações para aumentar as oportunidades e reduzir ameaças aos objetivos do projeto.
		\item Controlar os riscos e monitora-los durante o ciclo de vida do projeto.

	\end{itemize}

% section processo_de_gerenciamento_de_riscos (end)
\section{Responsabilidade dos Riscos da Equipe do Projeto} % (fold)
\label{sec:responsabilidade_dos_riscos_da_equipe_do_projeto}

Os processos de gerenciamento de riscos serão realizados pelo Scrum master de projeto, durante o período, no entanto todos os membros da equipe de desenvolvimento do R2-I2 serão consultados nesse processo para o levantamento de riscos dos sistemas e subsistemas pelos quais estão responsáveis, assim como as formas de controle dos mesmos.

\section{Probabilidade e Impacto de Riscos} % (fold)
 \label{sec:probabilidade_e_impacto_de_riscos}
 
 Diferentes riscos possuem diferentes probabilidades de ocorrência e diferentes impactos no projeto. Tendo isso em vista, foram feitas uma matriz de risco e probabilidade e uma matriz de impacto para auxiliar da qualidade e credibilidade da análise dos riscos assim como na decisão de respostas e plano de controle.

 \subsection{Matriz de Risco e Probabilidade} % (fold)
 \label{sub:matriz_de_risco_e_probabilidade}

 A tabela \ref{tab:probabilidade} mostra a matriz de probabilidade dos riscos com uma pontuação para a análise qualitativa.

 \begin{table}[H]
	\centering
	\caption{Matriz de probabilidade de riscos do R2-PI2.}
	\label{my-label}
	\begin{tabular}{|c|c|}
		\hline
		\rowcolor[HTML]{C0C0C0} 
		\textit{\textbf{Probabilidade}} & \textit{\textbf{\% de certeza}} \\ \hline
		1- Muito baixa                  & 0 a 20\%                        \\ \hline
		2- Baixa                        & 20 a 40\%                       \\ \hline
		3- Média                        & 40 a 60\%                       \\ \hline
		4- Alta                         & 60 a 80\%                       \\ \hline
		5- Muito alta                   & \textgreater80\%                \\ \hline
	\end{tabular}
\end{table}
 
 % subsection matriz_de_risco_e_probabilidade (end)
 \subsection{Matriz de Impacto dos Riscos} % (fold)
 \label{sub:matriz_de_impacto_dos_riscos}
 
 Para chegar em uma nota final de impacto para o risco, foram considerados 4 aspectos principais: Custo, Tempo, Escopo e Qualidade.
 % subsection matriz_de_impacto_dos_riscos (end)
 % section probabilidade_e_impacto_de_riscos (end) 


% section responsabilidade_dos_riscos_da_equipe_do_projeto (end)