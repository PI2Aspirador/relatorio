\section{Sensoriamento} % (fold)
\label{sec:sensoriamento}
\subsection{Contextualização}

Diante do projeto proposto, a equipe de eletrônica a princípio se propõe a desenvolver parte da automação e do controle do aspirador de pó, para isso o projeto eletrônico foi dividido em 3 partes, sendo elas:
  \begin{itemize}
    \item Instrumentação.
    \item Comunicação.
    \item Controle.
  \end{itemize}
% section contextualização (end)

\subsection{Instrumentação} % (fold)
\label{sub:instrumentação}

% subsection instrumentação (end)

A parte de instrumentação tem como principal objetivo solucionar o problema de colisão indesejada com a parede e outros móveis presentes na casa, para isso se faz necessário o uso de alguns componentes relativamente simples porem com grande aplicação, são eles:
 
  \subsubsection{Sensor de distância por infravermelho} 
  \label{sub:Sensor_de_distância_por_infravermelho}
    É possível utilizar um par formado por um LED emissor e um receptor de infravermelho para a detecção de obstáculos em robótica, como descrito por Lee e Chong (2011).  Essa solução será adotada no presente projeto para a detecção de objetos no trajeto do robô e segurança do protótipo \cite{detectar_objeto}.

    A solução foi escolhida devido a facilidade de obtenção dos componentes e baixo custo além da possibilidade de projetar o sensor ao invés de realizar a compra do módulo pronto, diminuindo mais ainda os custos de produção do robô aspirador.

    O circuito da figura abaixo serve para validar o sistema de detecção de obstáculos.

    \begin{figure}[H]                                       
      \centering                                            
      \includegraphics[scale=0.3]{figuras/sensor_proximidade.png} 
      \caption{Circuito de um sensor de proximidade usando LED emissor de infravermelho e um fototransistor.}    
      \label{img:sensor_proximidade}                                
    \end{figure}                                            

    A luz infravermelha gerada pelo LED emissor é refletida pelo objeto onde ela incide e atinge o fototransistor receptor que entra na sua zona de condução. Dependendo da quantidade de luz refletida de volta para o fototransistor, ele detecta o objeto a frente. 

    \begin{figure}[H]                                                           
      \centering                                                                
      \includegraphics[scale=0.5]{figuras/funcionamento_sensor_obstaculo.png}               
      \caption{ Esquema de funcionamento do sensor de obstáculos por IR (infravermelho).}    
      \label{img:esquema_sensor_proximidade}                                            
    \end{figure}         

    Através do circuito então, quando um objeto se aproximar do sensor, a uma distância mínima que será especificada, um sinal será emitido para o controlador que irá acionará os motores para desviarem do obstáculo, para garantir a eficiência dessa solução serão utilizados no mínimo 5 sensores, 3 na parte inferior de forma a analisar a frente e as laterais do robô, um na parte superior para detectar a proximidade de obstáculos altos como mesas, e um abaixo do aspirador para indicar um possível desnivelamento a fim de evitar uma queda do mesmo.
                               
  \subsubsection{Demultiplexador}                        
  \label{sub:demultiplexador}
    Visto que a quantidade de pares IR presente no projeto é maior que 5 se faz necessário o uso de uma estratégia de alimentação para não sobrecarregar o microprocessador. Optou-se então por utilizar um demultiplexador possibilitando assim o chaveamento dos emissores de IR através do clock do microprocessador.

    O componente escolhido para esse projeto foi o 4051, circuito integrado CMOS, consiste num Multiplexador/Demultiplexador de 8 canais que pode trabalhar tanto com sinais analógicos como digitais.\cite{newtoncbraga}

  \begin{figure}[H]                                                           
    \centering                                                                
    \includegraphics[scale=0.2]{figuras/multiplexador.png}             
    \caption{ CI CMOS 4051 multiplexador/demultiplexador de 8 canais.} 
    \label{img:multiplexador}                                            
  \end{figure}                                                                

  \subsubsection{Medidor de Bateria}
  \label{sub:Medidor_de_bateria}
    O aspirador de pó não pode ficar sem bateria no meio da execução da limpeza. Para conferir o nível de bateria, será utilizado um comparador de tensão. De forma que, assim que o medidor verificar que o nível da bateria está ficando muito reduzido, um sinal será enviado ao microprocessador que irá interromper o ciclo de limpeza e enviar o robô para a base de carregamento.

    Escolheu-se um circuito comparador que irá comparar a tensão da bateria com até 10 níveis de referência determinados de acordo com a tensão máxima da bateria e o mínimo aceitável para que o robô retorne à base. Essa comparação irá ocorrer por meio de amplificadores operacionais (Amp Op) que irão comparar a tensão que está vindo da bateria com aos níveis de referência. A saída do Amp Op será 1 se o sinal da bateria for maior do que a referência daquele amplificador ou será 0 se o sinal da bateria for menor do que a referência. Para tornar visível o nível da bateria, na saída de cada comparador será adicionado um LED.

    Além de acender ou não os LEDs, cada saída do comparador vai enviar um sinal para o controlador informando o nível da bateria para que ele possa interromper o ciclo de limpeza e enviar o robô para base.

% section instrumentação (end)

\subsection{Hardware para Comunicação}
\label{sub:Hardwar_para_Comunicação}
  A parte de comunicação tem como objetivo coletar informações vindas dos sensores já apresentados, interpreta-las e envia-las para a base onde é feito o intefaceamento com o usuário através de um aplicativo para celular, para essa comunicação serão utilizados 2 microprocessadores, são eles:

    \subsubsection{Arduíno}
    Para controlar corretamente as informações obtidas por todos os sensores da parte de instrumentação do projeto é necessário um processador com uma grande quantidade de portas analógicas e digitais, sendo assim utilizaremos o Arduíno Mega, que possui as seguintes características:

    \begin{figure}[H]                                                  
      \centering                                                       
      \includegraphics[scale=0.4]{figuras/arduino_mega.png} 
      \caption{Arduíno Mega.}                  
      \label{img:arduino_mega}                                             
    \end{figure}                                                       

    \begin{itemize}
      \item Microcontrolador: ATmega2560
      \item Voltagem de Alimentação: 5V
      \item Voltagem de entrada (recomendada): 7-12V
      \item Voltagem de entrada (limites): 6-20V
      \item Pinos digitais I/O: 54 (dos quais 14 podem ser saídas PWM)
      \item Pinos de entrada analógica: 16
      \item Corrente contínua por pino I/O: 40 mA
      \item Corrente contínua para o pino 3.3V: 50 mA
      \item Memória flash: 256 KB com 4 KB usado para bootloader
      \item SRAM: 8 KB
      \item EEPROM: 4 KB
      \item Velocidade de Clock: 16Mhz
    \end{itemize}
  
  \subsubsection{Módulo de WiFi}
  \label{sub:Modulo_de_wifi}
    Para que o Arduíno possa enviar as informações coletadas para a base é necessário que o mesmo se conecte a rede sem fio, para isso utilizaremos o módulo WiFi ESP8266 ilustrado na figura abaixo.

  \begin{figure}[H]                                      
    \centering                                           
    \includegraphics[scale=0.8]{figuras/modulo_wifi.png}
    \caption{Módulo WiFi ESP8266}                              
    \label{img:modulo_wifi}                             
  \end{figure}                                           

  \subsubsection{Raspberry Pi}
  \label{sub:Raspberry}
    Na base para o tráfego de dados, a placa escolhida foi a Raspberry Pi 2 modelo B, este modelo apresenta:

  \begin{figure}[H]                                      
    \centering                                           
    \includegraphics[scale=0.8]{figuras/rasp.jpg} 
    \caption{Raspberry Pi 2 B.}                        
    \label{img:Rasp}                              
  \end{figure}                                           

  \textbf{Especificações:}
  \begin{itemize}                                                     
    \item Chip: Broadcom BCM2836 SoC
    \item Arquitetura: Quad-core ARM Cortex-7
    \item CPU: 900Mhz
    \item Memória RAM: 1GB
    \item GPU Broadcom VideoCore IV
    \item Tensão de operação: Micro USB socket 5V/2ª
    \item Dimensões: 85 x 56 x 17 mm
 \end{itemize}                                                       
  
  \textbf{Conectores:}
  \begin{itemize}                                                     
    \item 4 portas USB
    \item GPIO de 40 pinos
    \item Full HDMI
    \item Ethernet 10/100 (RJ45)
    \item Saída de vídeo via HDMI, Composite (PAL e NTSC) ou Raw LCD (DSI)
    \item Saída de áudio via conector de 3,5mm
    \item Camera interface (CSI)
    \item Slot MicroSD
    \item VideoCore IV 3D graphics core
  \end{itemize}

  Com esse modelo, é possível realizar conexões com a internet e enviar informações ao usuário ou a um banco de dados, por exemplo.

\subsection{Controle} % (fold)
	\label{sub:controle}

		\subsubsubsection{Sistema de Controle do R2-PI2}

			A parte de controle tem como objetivo desenvolver um sistema capaz de monitorar e controlar a movimentação do aspirador para garantir que os motores sejam devidamente alimentados e juntamente com a parte de instrumentação e comunicação possa garantir a locomoção do robô durante a limpeza sem riscos de colisão com obstáculos.

			O controlador por ora escolhido, Arduino, irá controlar o funcionamento dos motores carrinho e dos sensores. Abaixo encontra-se o diagrama de blocos para o sistema em malha fechada do controle do R2-P12.

			\begin{figure}[H]
				\centering
				\includegraphics[scale=0.55]{figuras/diagrama_blocos_R2PI2.png}
				\caption{Diagrama de blocos do sistema de controle do R2-PI2. Fonte \cite{mello}.}
				\label{img:diagrama_sistema_controle}
			\end{figure}

			Como mostrado no diagrama, o controlador irá enviar sinais para os motores permitindo que eles sejam ligados ou desligados, fazendo o carrinho se movimentar de acordo com os parâmetros medidos pelos sensores infravermelhos (IR). 

			Serão utilizados sensores IR configurados como detectores de proximidade, na parte frontal  e lateral do aspirador, evitando possíveis colisões com obstáculos em seu caminho, e embaixo dele para evitar vãos como escadas e impedir a queda do robô. Encoderes serão utilizados como controle de posição e também no auxílio da movimentação das rodas quando o aspirador girar para desviar de obstáculos. O monitoramento será contínuo, portanto sempre que houver algum obstáculo ao alcance dos sensores , o controlador enviará um sinal para os motores, impedindo que ocorra choques com objetos no caminho. 

			Será monitorado também, o nível de bateria do robô, quando ele estiver abaixo de um limite que será estabelecido, o controlador enviará um sinal para que o robô possa então retornar para a base e carregar sua bateria. Os componentes envolvidos nessa área são apresentados a seguir.

			\begin{enumerate}
				\item \textbf{Ponte H}:

					Para que se possa controlar a direção e a velocidade dos motores , é necessário que a corrente elétrica possa fluir nas duas direções dentro de sua bobina, gerando campos magnéticos com intensidade e sentidos opostos. A configuração mais utilizada para controlar a corrente nesse projeto é um driver em ponte H (INOUE e OSUKA, 2004).

					O circuito da Ponte H é constituído por quatro transistores que atuam como chave e que, dependendo da configuração do chaveamento, determinam o sentido de rotação dos motores, como pode ser observado na Figura \ref{img:configH}.

					\begin{figure}[H]
						\centering
						\includegraphics[scale=0.3]{figuras/configH.png}
						\caption{Configurações da Ponte H.}
						\label{img:configH}
					\end{figure}

					A princípio a ponte H escolhida segue o modelo apresentado na Figura \ref{img:modeloH}. 

					\begin{figure}[H]
						\centering
						\includegraphics[scale=0.6]{figuras/modeloH.png}
						\caption{Ponte H.}
						\label{img:modeloH}
					\end{figure}

					A escolha foi feita baseando-se na compatibilidade do drive com os motores que devem ser utilizados na movimentação do aspirador.
				\item \textbf{PWM}:

					O controle das velocidades dos motores será feito por meio de chaveamento em frequência constante, gerado pelas saídas digitais do controlador. Para isso utiliza-se o conceito de Pulse-Width Modulation (Modulação por largura de pulso), ou PWM. Com uma onda quadrada com frequência constante e razão cíclica (duty cycle) ajustável , é possível transferir uma determinada quantidade de potência desejável através do valor médio de tensão do sinal \cite{ahmedi}.

					Segundo \cite{ahmedi}, a tensão média de saída é dada por

					\begin{equation}\label{3}
						V_{0} = \frac{Ton.vi}{T}
					\end{equation}
					Onde  é a tensão média de saída,  é o período em segundos em que o sinal fica em nível alto, T o período total do sinal e Vi a tensão de nível alto.

					A potência de saída do sinal pode ser descrita como:

					\begin{equation}
					P = V_{0} . I_{0}
					\end{equation}

					Sendo P a potência de saída,  a tensão de saída e  a corrente de saída. Portanto a partir das equações anteriores pode-se afirmar que:

					\begin{equation}
					V_{0} = Vi . d
					\end{equation}

					onde:

					$d = \frac{Ton}{T}$

					sendo  o duty cycle. A partir da lei de Ohm tem-se então que :

					\begin{equation}
					I_{0} = \frac{d . Vi}{R}
					\end{equation}

					E a potência de saída é dada por:

					\begin{equation}
					P_{0} = \frac{(D . Vi)^{2}}{R}
					\end{equation}

					Considerando-se uma carga totalmente resistiva, pode-se controlar a potência entregue de maneira proporcional a largura de pulso ao quadrado.

					Deve-se observar a frequência de trabalho, para não ultrapassar os limites do hardware de potência em casos onde há carga indutivas, como motores, é necessário pulsar uma frequência que faça a corrente estável para uma mesma largura de pulso, suavizando assim o movimento do motor.

				\item \textbf{Encoder}:

					Os encoders são sensores acoplados no motor para medir a velocidade e posição angular de acordo com a rotação sensoriada por ele. A especificação dos encoders absolutos são medidas em contagem por rotação, CPR, pois de acordo com o número de divisões do encoder e o tamanho da roda conhecem as velocidades, angular e linear, e também a trajetória percorrida \cite{braga}.

					\begin{figure}[H]
						\centering
						\includegraphics[scale=0.6]{figuras/encoder.png}
						\caption{Sistema de funcionamento de um encoder absoluto.}
						\label{img:encoder}
					\end{figure}

					A posição relativa do aspirador de pó é uma variável muito importante para a navegação e localização que pode ser medida através do uso de encoders [3]. Fixado junto ao motor, o encoder rotacional irá medir a quantidade de rotações do motor e, portanto, possibilitar a obtenção de informações sobre a velocidade angular e posição relativa. 

					Para esse projeto escolheu-se o módulo encoder P17 \ref{img:modulo_encoder} que realiza a leitura do sentido do giro e pode ser integrado ao Arduíno.

					\begin{figure}[H]
						\centering
						\includegraphics[scale=0.6]{figuras/modulo_encoder.png}
						\caption{Módulo encoder.}
						\label{img:modulo_encoder}
					\end{figure}

			\end{enumerate}



	% subsection controle (end)
% section instrumentação (end)