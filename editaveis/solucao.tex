Primeiramente começamos com a apresentação da solução proposta (tarefa que está com o Jair), sem apresentar nenhum detalhe técnico nem nada parecido, apenas explicando o funcionamento da solução. 

Esta solução está organizada segundo a EAP (Estrutura analítica do projeto) apresentada na figura x, destacando os entregáveis e seus sub-sistemas ao longo de todo o projeto.

FIGURA DA EAP AQUI!

\subsection{Requisitos do sistema} % (fold)
\label{sub:requisitos_do_sistema}

% subsection requisitos_do_sistema (end)
\subsection{Alimentação} % (fold)
\label{sub:alimentação}
	Definição detalhada da solução em relação a alimentação (Luan)
% subsection alimentação (end)

\subsection{Navegação} % (fold)
\label{sub:automação}
	Definição detalhada da solução em relação a navegação. (João Paulo)
% subsection automação (end)

\subsection{Estrutura} % (fold)
\label{sub:alimentação}
	Definição detalhada da solução em relação a estrutura. (Rafael)
% subsection alimentação (end)

\subsection{Sucção} % (fold)
\label{sub:aspirador}
	Definição detalhada da solução em relação a sucção. (Yago)
% subsection aspirador (end)

\subsection{Instrumentação} % (fold)
\label{sub:instrumentação}
	Definição detalhada da solução em relação a instrumentação. (Kaio)
% subsection instrumentação (end)

\subsection{Comunicação} % (fold)
\label{sub:comunicação}
	Definição detalhada da solução em relação a comunicação. (Fazzolino)
% subsection comunicação (end)

\subsection{Interface} % (fold)
\label{sub:interface}
	Definição detalhada da solução em relação a interface. (Ricardo)
% subsection interface (end)

\subsection{Locomoção} % (fold)
\label{sub:locomoção}
	Definição detalhada da solução em relação a locomoção. (Yago)
% subsection locomoção (end)