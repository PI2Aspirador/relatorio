Primeiramente começamos com a apresentação da solução proposta (tarefa que está com o Jair), sem apresentar nenhum detalhe técnico nem nada parecido, apenas explicando o funcionamento da solução. 

Esta solução está organizada segundo a EAP (Estrutura analítica do projeto) apresentada na figura x, destacando os entregáveis e seus sub-sistemas ao longo de todo o projeto.

FIGURA DA EAP AQUI!

\section{Requisitos do sistema} % (fold)
\label{sub:requisitos_do_sistema}

% section requisitos_do_sistema (end)
\section{Alimentação} % (fold)
\label{sub:alimentação}
	Definição detalhada da solução em relação a alimentação (Luan)
% section alimentação (end)

\section{Navegação} % (fold)
\label{sub:automação}
	A solução referente à navegação possui enfoque principal no algoritmo de controle que trabalhará a trajetória a ser percorrida, a identificação de obstáculos e replanejamento da trajetória e a lógica de controle para retorno à base. Diversas possibilidades foram estudadas e analisadas, desde o planejamento de rota partindo com formato espiral, até trajetórias aleatórias com simples desvios de obstáculos.

	A partir e comparações e estudo de concorrentes, como o robô \textit{Roomba}\footnote{https://www.irobot.com.br/}, por exemplo, observou-se que na grande maioria, o sistema de navegação escolhido pelos fabricantes é baseado em trajetórias aleatórias. Segundo \cite{robo_limpeza_domesti}, a utilização de navegação aleatória garante um bom desempenho, já que com o passar do tempo, o robô consegue acessar o cômodo como um todo. Dessa forma, optou-se pela utilização de um algoritmo de navegação baseado em trajetória aleatória para o desenvolvimento do sistema \textit{R2-PI2}.

	Um dos grandes problemas encontrados na navegação é referente a volta à base por parte do robô. Identificar onde a base se encontra e traçar uma rota até a mesma é uma tarefa que necessita de algumas ferramentas, como a utilização de sinais e sensores para comunicação entre a base e o robô. Para isso, optou-se pela utilização de 3 (três) sensores infra-vermelho emitidos pela base com uma angulação de 45º entre eles, fazendo com que o robô possa identificar o sinal e navegar até sua fonte, a base.
% section automação (end)

\section{Estrutura e Locomoção do robô} % (fold)
\label{sub:alimentação}
	
	O robô, de forma autônoma, deve percorrer todo o cômodo escolhido para limpá-lo para isso deve ser desenvolvido todo o sistema de locomoção dele, incluindo as rodas, motor, caixa de redução e todo o estudo dinâmico relacionado. Foram cogitadas duas soluções para a estrutura do robô e sua forma de locomoção, de forma que a segunda solução foi escolhida. A escolha da segunda solução se justifica com base nos requisitos do sistema de sensoriamento quanto a posição de sensores pela estrutura e devido ao maior erro propagado pela primeira solução na navegação inercial do sistema. 

	\subsection{Solução 1} % (fold)
	\label{sub:solução_1}

		O robô seria composto de uma chapa retangular de alumínio de 400mm x 300mm x 5mm, que servirá como base para a distribuição e união dos componentes do projeto. Essa base retangular poderá ser usinada para outra forma caso haja necessidade de diminuição do tamanho do robô. O material escolhido para a base foi o alumínio pela sua leveza, resistência e preço. Sua área  de superfície é capaz de abrigar todos os componentes do aspirador e ainda possui espaços para criar novas soluções ou até mesmo de componentes para a refrigeração dos subsistemas do robô.

		O sistema de tração dessa base será feito por uma esteira tipo lagarta, muito utilizada em tanques de guerra. Esse sistema é bastante robusto e aguentaria o peso de todo a estrutura sem problemas. Isso tiraria a necessidade de colocar um sistema de suspensão.

		A estrutura do aspirador e seus componentes:

		\begin{itemize}
			\item 2 motores com caixa de redução ligados a chapa de alumínio;
			\item Os eixos que ligaram as rodas a chapa de alumínio serão feitos com aço e serão usinados nas pontas para fazer roscas que irão fixar as rodas;
			\item 4 engrenagens grandes que serão utilizadas como rodas;
			\item 2 engrenagens menores que irão ser ligadas direto nos dois motores;
			\item Conectores múltiplos, do tipo que se usa em chuveiros para ligar os eixos na chapa de alumínio;
			\item Correntes de bicicleta que ligaram as engrenagens e farão o papel de esteira.
		\end{itemize}

		Entre esses dois sistemas foi escolhido o segundo pela sua construção ser mais robusta e suporta mais os esforços que será submetido o robô. Um grande problema do primeiro sistema é que a sustentação da estrutura se daria no próprio eixo do motor, que é de plastico, o que poderia causar a quebra do sistema, já no segundo sistema a sustentação é feita nos eixos, que são feitos de aço.
		
		\begin{figure}[H]
			\centering
			\includegraphics[scale=0.8]{figuras/rascunho_estrutura.png}
			\caption{Rascunho da estrutura - Solução 1}
			\label{img:rascunho1}
		\end{figure}

		\subsubsection{Orçamento da solução 1}

			As aquisições necessárias estão listadas abaixo:
			\begin{itemize}
				\item 2 motores com caixa de redução = R\$ 74 ,00
				\item 2 roldanas pequenas= R\$5,00
				\item 4 roldanas grandes = R\$ 16,00
				\item Corrente de bicicleta = R\$10,00
				\item Parafusos = R\$20,00
				\item Multi fixadores = R\$ 5,00
			\end{itemize}

		\subsection{Solução 2} % (fold)
		\label{sub:solução_2}
			
			O robô aspirador terá forma circular. Esse formato foi escolhido para facilitar as manobras de curvas, aumentando a área que ele irá percorrer. Outra vantagem que esse formato fornece é a questão do controle autônomo dele, assim facilita a distribuição dos sensores e o próprio controle do movimento do robô, pois resulta em menos erros. A estrutura do robô deve ser tal para suportar as cargas dos equipamentos do interior do robô como os sensores, motores, coolers e o sistema de sucção sem que sofra deformações. Além dessas forças deve-se também ser resistente à fadiga, já que estará sujeito a cargas contínuas e repetidas, e a impactos contra objetos ou paredes. Um material que já é utilizado em muitas aplicações pois apresenta boa propriedades é o Alumínio. A tabela a seguir mostra alguns valores das propriedades mecânicas do alumínio.

			\begin{figure}[H]
				\centering
				\includegraphics[scale=0.8]{figuras/tabela_estru1.png}
				\caption{Propriedades mecânicas do alumínio.}
				\label{img:rascunho1}
			\end{figure}

			Então será construída uma base circular de alumínio de 40 cm de diâmetro.

        	Com relação a movimentação do robô 3 rodas serão suficientes para garantir o equilíbrio. Duas rodas serão tracionadas uma livre.  A roda livre é do tipo esfera e as outras duas serão de um kit motor redução, que junto à roda está montado o motor com uma caixa de redução para aumentar o torque. A mudança de direção e giro do robô é realizada alternando a potência fornecida em cada roda ou invertendo o sentido de rotação, por exemplo para fazer com que ele gire para a esquerda, deve diminuir a potência da roda esquerda e manter a potência da roda direita. As figuras seguintes ilustram as rodas e motores utilizados.

        	\begin{figure}[H]
				\centering
				\includegraphics[scale=0.7]{figuras/motor_roda.png}
				\caption{Kit motor redução. Disponível em http://www.huinfinito.com.br/motores/787-motor-com-engrenagem-de-reducao-3-6v-90-graus.html?search\_query=roda\&results=7}
				\label{img:kit_motor}
			\end{figure}

			\begin{figure}[H]
				\centering
				\includegraphics[scale=0.7]{figuras/esfera.png}
				\caption{Roda do tipo esfera. Disponível em http://produto.mercadolivre.com.br/MLB-772876471-roda-boba-robot-caster-esfera-roda-rodinha-pra-robotica-robo-\_JM}
				\label{img:esfera}
			\end{figure}

			As especificações da roda e do motor são mostradas na tabela \ref{tab:motor_red}:

			\begin{table}[H]
				\centering
				\caption{Especificação motor de redução}
				\label{tab:motor_red}
				\begin{tabular}{|c|c|}
					\hline
					\multicolumn{2}{|c|}{\cellcolor[HTML]{C0C0C0}\textbf{Especificação Motor}} \\ \hline
					\textit{\textbf{Tamanho}}                            & 69x37x22,7mm        \\ \hline
					\textit{\textbf{Peso}}                               & 29g                 \\ \hline
					\textit{\textbf{Formato}}                            & 90 graus            \\ \hline
					\textit{\textbf{Tensão de operação}}                 & 3 a 6V              \\ \hline
					\textit{\textbf{Relação de transmissão}}             & 1:120               \\ \hline
					\textit{\textbf{Velocidade a 3V(sem carga)}}         & 100 rpm             \\ \hline
					\textit{\textbf{Corrente a 3V(sem carga)}}           & 60 mA               \\ \hline
					\textit{\textbf{Corrente a 3V(com carga)}}           & 260 mA              \\ \hline
					\textit{\textbf{Torque a 3V}}                        & 1.20 kgf-cm         \\ \hline
					\textit{\textbf{Velocidade a 6V(sem carga)}}         & 200 rpm             \\ \hline
					\textit{\textbf{Corrente a 6V(sem carga)}}           & 71 mA               \\ \hline
					\textit{\textbf{Corrente a 6V(com carga)}}           & 470 mA              \\ \hline
					\textit{\textbf{Torque a 6V}}                        & 1.92 kgf-cm         \\ \hline
					\textit{\textbf{Diâmetro externo do eixo}}           & 5,4 mm "I"          \\ \hline
				\end{tabular}
			\end{table}

			\begin{figure}[H]
				\centering
				\includegraphics[scale=0.5]{figuras/estrutura_circular.png}
				\caption{Estrutura circular de integração dos subsistemas.}
				\label{img:estrutura_circular}
			\end{figure}

			\begin{table}[H]
				\centering
				\caption{Especificações roda tracionada}
				\label{tab:especificacoes_roda}
				\begin{tabular}{|c|c|}
					\hline
					\multicolumn{2}{|c|}{\cellcolor[HTML]{C0C0C0}\textbf{Especificação da Roda}}                 \\ \hline
					\textit{\textbf{Material}}                             & Roda plástica com pneu de borracha. \\ \hline
					\textit{\textbf{Diâmetro externo}}                     & 65 mm                               \\ \hline
					\textit{\textbf{Largura pneu}}                         & 26 mm                               \\ \hline
					\textit{\textbf{Diâmetro interno para engate do eixo}} & 5,4 mm "I"                          \\ \hline
				\end{tabular}
			\end{table}

			A parte superior da estrutura, ou seja, a tampa, será fabricada em PVC ou em acrílico. A escolha de um material plástico deve-se a facilidade de manuseio, facilitando molda-lo à forma desejada. Deixa a estrutura mais leve, fazendo com que o motor realize menos trabalho, e pode suportar valores altos de cargas, resistindo a impactos. O material acrílico (METACRILATO DE METILA) possui (densidade relativa de 1.19 g/cm3), resistente a água e boa resistência segundo a Figura \ref{img:acrilico}:

			\begin{figure}[H]
				\centering
				\includegraphics[scale=0.8]{figuras/tabela_acrilico.png}
				\caption{Valores de propriedades mecânicas do Acrílico. Adaptado de http://www.indac.org.br/arquivos/acrilico\_indac.pdf}
				\label{img:acrilico}
			\end{figure}

			Na tabela \ref{tab:pvc} são apresentados os valores referentes ao PVC.

			\begin{table}[H]
				\centering
				\caption{Valores das propriedades mecânicas do PVC}
				\label{tab:pvc}
				\begin{tabular}{|c|c|c|c|}
					\hline
					\textit{\textbf{Materiais}} & \textit{\textbf{\begin{tabular}[c]{@{}c@{}}Resistência a tração\\ (N/mm\textsuperscript{2})\end{tabular}}} & \textit{\textbf{\begin{tabular}[c]{@{}c@{}}Módulo de elasticidade\\ (kN/mm\textsuperscript{2})\end{tabular}}} & \textit{\textbf{\begin{tabular}[c]{@{}c@{}}Densidade\\ (kg/m\textsuperscript{3})\end{tabular}}} \\ \hline
					PVC                         & 55                                                                                       & 3.5                                                                                         & 1400                                                                          \\ \hline
				\end{tabular}
			\end{table}

			Ambos os materiais apresentam boa resistência à tração e podem ser aplicados ao projeto e irão proteger os circuitos, baterias, motores e outros equipamentos sensíveis em seu interior. A placa de acrílico ou de PVC cortado e usinado para se encaixar na estrutura montada utilizando parafusos e porcas. Os parafusos irão facilitar o trabalho de encaixe e desencaixe da tampa de acrílico para ajustes e limpezas das peças, além de fixar melhor. Se a tampa fosse colada na estrutura não haveria essa possibilidade.

			\subsubsection{Orçamento}

			\begin{itemize}
				\item 2 Kit motor redução – R\$ 49,80
				\item 1 roda esférica – R\$ 17,95
				\item Chapa de alumínio – R\$ 24,90 
				\item Chapa de PVC – 220cm x 110cm x 10mm - R\$ 17,00
				\item Chapa de acrílico 100cm x 50cm x 2mm – R\$ 54,90
			\end{itemize}
		% subsection solução_2 (end)

	% subsection solução_1 (end)
% section alimentação (end)

\section{Estrutura da Base} % (fold)
\label{sec:estrutura_da_base}
	
	O robô terá uma base de recarga automática responsável pelo guiamento do sistema pelo ambiente, pelo recarregamento da bateria e por abrigar e alimentar o Raspberry Pi que fará os cálculos de movimento do sistema.

	\subsection{Solução} % (fold)
	\label{sub:solução}
		
		O requisito é que o robô ao identificar que está com bateria baixa irá seguir para a base seguindo o sinal emitido por ela. A estrutura da base não necessita ter grande porte, por ser fixa e possuir menos equipamentos em seu interior. A base terá uma carcaça quadrada de dimensões 250x250x200mm, como se fosse uma caixa e assim como o robô aspirador, será feito em acrílico ou PVC pela leveza, resistência e custo. Como é uma peça de plástico também evitará condução de corrente, mantendo a proteção do usuário contra choques.  O conector será do tipo magnético, pois no momento em que for ocorrer o encaixe entre as peças do conector, esta possa ser feita de modo mais certeiro e ficará na face oposta à que fica apoiada na parede.

		\begin{figure}[H]
			\centering
			\includegraphics[scale=0.8]{figuras/conector_mag.png}
			\caption{Conectores magnético modelo Sony Xperia.}
			\label{img:conectores}
		\end{figure}

		\begin{figure}[H]
			\centering
			\includegraphics[scale=0.8]{figuras/estrutura_base.png}
			\caption{Estrutura da base de recarga.}
			\label{img:estrutura_base}
		\end{figure}	

	\subsection{Orçamento} % (fold)
	\label{sub:orçamento}
		
		\begin{itemize}
			\item Chapa de alumínio – R\$ 24,90 
			\item Chapa de PVC – 220cm x 110cm x 10mm - R\$ 17,00
			\item Chapa de acrílico 100cm x 50cm x 2mm – R\$ 54,90
			\item Conectores de carga Xperia Z3 – R\$ 55,00
		\end{itemize}	
	% subsection orçamento (end)	

	% subsection solução (end)
% section estrutura_da_base (end)

\section{Sucção} % (fold)
\label{sub:aspirador}
	Definição detalhada da solução em relação a sucção. (Yago)
% section aspirador (end)

\section{Instrumentação} % (fold)
\label{sub:instrumentação}
	Definição detalhada da solução em relação a instrumentação. (Kaio)
% section instrumentação (end)

\section{Comunicação} % (fold)
\label{sub:comunicação}
	De acordo com o apresentado nesta documentação de projeto, a solução proposta envolve diversos módulos funcionais, como o robô em si, a base fixa, e o sistema de controle. A partir de uma visão de alto nível do projeto, como a apresentada na Figura \ref{img:arq_comu}, é possível observar de maneira clara os módulos que deverão se comunicar para garantir o funcionamento do sistema como um todo.

	\begin{figure}[H]
		\centering
		\includegraphics[scale=0.8]{figuras/arquitetura_comunicacao.png}
		\caption{Arquitetura de comunicação do sistema}
		\label{img:arq_comu}
	\end{figure}

	Com o objetivo de garantir que a solução seja confiável e resistente a situações críticas, como a falta de internet, por exemplo, o sistema foi planejado para disponibilizar uma sub-rede interna, tendo como fonte a base de recarga do robô. Cada módulo que necessita de comunicação deverá se conectar a rede, viabilizando o funcionamento do sistema mesmo em momentos com falha de internet, já que todos os envolvidos compartilham o mesmo ambiente.

	A comunicação entre o robô (\textit{arduino}) e a base (raspberry) fará uso desta rede wifi, a partir da utilização de um módulo wifi, o \textit{arduino} poderá acessar a rede, possibilitando a comunicação via tcp/ip. Já a \textit{raspberry} se conectará a rede via cabo \textit{ethernet}.

	O roteador utilizado é da familia D-link, seguindo o protocolo de certificação WPA2, que utiliza o EAS (Advanced Encryption Standard), como sistema de encriptação. Segundo \cite{wpa2}, este protocolo possui uma confiabilidade bem maior que a encontrada em seu antecessor, WPA. Ainda de acordo com \cite{wpa2}, este sistema de segurança envolve um algoritmo de criptografia robusto, utilizando chaves de 128 a 256 bits maximizando a segurança da rede.


	O núcleo da rede, ou seja, o ponto central da comunicação do sistema se encontra na base de recarga do robô, que está detalhada no tópico a seguir.

	\subsection{Base de regarga}

	A base de recarga do robô sustentará todo o sistema de inteligência da solução, assim como a sub-rede que possibilitará a comunicação entre os módulos. Será utilizada uma \textit{raspberry PI} como servidor central do sistema, processando e controlando toda a solução. O servidor será implementado utilizando a tecnologia Ruby on Rails, no sistema operacional \textit{Raspbian} (Debian).

	Além da sustentação da \textit{raspberry}, é necessário sustentar um roteador D-link 524 para implementação e sustentação da rede wifi que será utilizada como meio de comunicação do sistema, e 3 (três) emissores infra-vermelho, utilizados para retorno do robô à base.
	
% section comunicação (end)

\section{Interface} % (fold)
\label{sub:interface}
	Definição detalhada da solução em relação a interface. (Ricardo)
% section interface (end)
