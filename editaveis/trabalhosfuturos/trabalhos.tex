\section{Trabalhos Futuros}

\subsection{Navegação}

Como dito na seção de integração, foi necessário, devido ao tempo de entrega, remover a funcionalidade de volta à base autônoma. Para realizar essa volta em trabalhos futuros, serão necessários filtros para os dados coletados de giro de motor e dados dos sensores de distância do robô e a base. Com esses dados, será utilizado o algoritmo implementado para descobrir onde se encontra a base em relação ao robô e girar confiavelmente o robô na direção da base. Após esse giro, o robô deverá seguir em linha reta até a base ou até encontrar um obstáculo. Após desviar do obstáculo, o robô começará o procedimento de descoberta da localização da base novamente.

\subsection{Sucção}

\begin{itemize}
   \item \textbf{Melhorar a construção geral da escova.}\\
    As cerdas da escova possuem muito atrito ao rodar devido a problemas de construção, alinhamento e fixação do eixo da cerdas até o motor. De uma maneira geral a construção está muito rústica e imprecisa.

    \item \textbf{Modificar a geometria das cerdas da escova.}\\
    Modificar a geometria das cerdas para reproduzir uma espiral de parafuso aumentaria a eficiência da escova na coleta de sujeira para seu compartimento interno.


  \item \textbf{Trocar o material utilizado para varrer a sujeira.}\\
  Trocar o pano utilizado por um material impermeável e mais resistente.


  \item \textbf{Isolar o eixo do motor da sujeira.}\\
  Atualmente a sujeira tem formas de chegar ao eixo do motor, o que pode eventualmente causar o travamento devido ao excesso de sujeira em volta do eixo.


  \item \textbf{Acrescentar um dissipador térmico passivo as baterias.}\\
  Devido à alta temperatura das baterias, recomenda-se acrescentar um sistema de dissipação térmica por motivos de segurança.

\end{itemize}
